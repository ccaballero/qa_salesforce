\chapter{Introducción}

Hoy en día muchas empresas de software, siguen procedimientos de desarrollo
ágiles, agilizando la producción y publicación de productos hacia el mercado,
esto presenta un gran reto tanto desde la perspectiva del aseguramiento de la
calidad del producto, como del seguimiento y control de la corrección de
errores.

La clave para agilizar estos procesos parte por optimizar y automatizar los
procesos de despliegue y evaluación del software, el cambio es necesario para
el manejo de proyectos de desarrollo grandes, o proyectos que siguen
lineamientos o estándares de cumplimiento muy estrictos.

Este capitulo define el propósito del presente trabajo, los antecedentes, los
objetivos del mismo, los alcances, y los factores que despertaron el interés
por resolverlos.

\section{Antecedentes}
\emph{Salesforce} es un servicio que apoya a la gestión de todo el flujo de
clientes que se tienen en una empresa, este tipo de software es  conocido como
\emph{CRM}, la meta principal es ayudar a gestionar las relaciones con los
clientes, conocer su necesidades y preferencias.

Entre los muchos componentes que integran el servicio, uno de los elementos mas
básicos es aquel que gestiona los productos y las listas de precios.

Los productos son un catálogo de base de todos los elementos y servicios que
pueden venderse y sus precios estándar. Las listas de precios permiten crear un
conjunto personalizado de productos con precios de lista asociados para usos
específicos, ambos integran múltiples flujos de trabajo dentro de
\emph{Salesforce}.

\section{Definición del problema}
Siendo el modulo de productos parte integrante de otros módulos de
\emph{Salesforce}, un potencial error en este podría tener un efecto en cascada
que afectaría a otros componentes, lo que conduce a dar mayor prioridad a la
evaluación de la funcionalidad disponible por este modulo.

Tanto el modulo de productos como el modulo de listas de precios, componen un
conjunto de elementos muy acoplados entre si, haciendo que ambos requieran ser
evaluados como una unidad única.

\emph{Salesforce} sigue un política de actualizaciones muy fluida, pudiendo
existir hasta tres versiones por año, lo que sin una automatización correcta
de las pruebas de software, implicaría un enorme gasto de recursos para la
compañía.

Por lo mencionado se define el problema como:

\emph{«Garantizar la calidad de los elementos que componen la interfaz de
usuario requiere de un proceso de evaluación continua, de forma que el
producto y sus diferentes versiones contengan la menor cantidad de errores.»}.

\section{Objetivos}

\subsection{Objetivo General}
Automatizar los casos de prueba de las funcionalidades provistas por
\emph{Salesforce} que componen la interfaz de usuario del modulo de gestión de
productos y listas de precios, para minimizar la cantidad de errores que
contiene el software.

\subsection{Objetivos Específicos}
\begin{itemize}
\item Formular los casos de prueba necesarios que el modulo de gestión de
    productos requiera para cubrir los atributos de calidad requeridos.
\item Automatizar los casos de prueba de las funciones que componen la interfaz
    de usuario del módulo de gestión de productos.
\item Formular los casos de prueba necesarios que el modulo de gestión de
    listas de precios requiera para cubrir los atributos de calidad requeridos.
\item Automatizar los casos de prueba de las funciones que componen la interfaz
    de usuario del módulo de gestión de lista de precios.
\end{itemize}

\section{Justificación}
Hoy en día, la automatización de pruebas es una tarea esencial para proporcionar
un servicio de testing adecuado ya que los sistemas han crecido tanto en tamaño
como en complejidad.

Es necesario tener el tiempo de probar las nuevas funcionalidades del sistema
sin ignorar la funcionalidad previa. Una estrategia de automatización
implementada apropiadamente nos ayudará a lograrlo junto con los beneficios
propios de la automatización \cite{Software}.

\section{Innovación tecnológica}
Se plantea la realización de este proyecto haciendo uso de la biblioteca
\emph{webdriver.io}, un framework de automatización de pruebas de interfaz de
usuario escrito en \emph{javascript}, para facilitar la ejecución de las
pruebas tanto en diferentes ámbitos de ejecución como en múltiples y variados
entornos de prueba.

\section{Alcance}
El proyecto cubrirá exclusivamente los componentes de interfaz del modulo de
productos y listas de precios, sin considerar aquellas funcionalidades dentro de
estos que requieran la utilización de otros módulos, de esta forma no se tomaran
en cuenta los siguientes aspectos:

\begin{itemize}
\item Generación de gráficas dentro de los módulos a evaluar.
\item Configuración de múltiples familias de productos.
\item Gestión de permisos de visibilidad de los elementos hacia diversas
    categorías de usuarios.
\end{itemize}

\section{Planificación de actividades}
Para conseguir los objetivos planteados por el proyecto sea realizaran las
actividades detalladas en el cuadro \ref{planificacion} en la página
\pageref{planificacion}.

\begin{sidewaystable}
\centering
\small
{\def\arraystretch{1.75}
\begin{tabular}{|l|l|p{6.5cm}|l|}
\hline
Objetivo General & Objetivos Específicos & Actividades & Resultados \\
\hline
\multirow{12}{4.0cm}{Automatizar los casos de prueba de las funcionalidades
provistas por \emph{Salesforce} que componen la interfaz de usuario del modulo
de gestión de productos y listas de precios, para minimizar la cantidad de
errores que contiene el software.} &
\multirow{3}{4.0cm}{Formular los casos de prueba necesarios que el modulo de
gestión de productos requiera para cubrir los atributos de calidad requeridos.} &
Recolectar información, explorar y analizar el módulo de productos. &
\multirow{3}{4.0cm}{Casos de prueba del modulo de productos.} \\
\cline{3-3}
& & Diseñar los tipos de evaluación requeridos para el módulo de productos. & \\
\cline{3-3}
& & Formular los casos de prueba necesarios para el módulo de productos. & \\
\cline{2-4}

& \multirow{3}{4.0cm}{Automatizar los casos de prueba de las funciones que
componen la interfaz de usuario del módulo de gestión de productos.} &
Analizar, y diseñar el software para la automatización de la interfaz de
usuario. &
\multirow{3}{4.0cm}{Rutinas de automatización de los casos de prueba del modulo
de productos.} \\
\cline{3-3}
& & Implementar el software para automatización de los casos de prueba. & \\
\cline{3-3}
& & Ejecutar el software para automatización de los casos de prueba. & \\
\cline{2-4}

& \multirow{3}{4.0cm}{Formular los casos de prueba necesarios que el modulo de
gestión de listas de precios requiera para cubrir los atributos de calidad
requeridos.} &
Recolectar información, explorar y analizar el módulo de listas de precios. &
\multirow{3}{4.0cm}{Casos de prueba del modulo de listas de precios.} \\
\cline{3-3}
& & Diseñar los tipos de evaluación requeridos para el módulo de listas de
precios. & \\
\cline{3-3}
& & Formular los casos de prueba necesarios para el módulo de listas de
precios. & \\
\cline{2-4}

& \multirow{3}{4.0cm}{Automatizar los casos de prueba de las funciones que
componen la interfaz de usuario del módulo de gestión de lista de precios.} &
Analizar, y diseñar el software para la automatización de la interfaz de
usuario. &
\multirow{3}{4.0cm}{Rutinas de automatización de los casos de prueba del modulo
de listas de precios.} \\
\cline{3-3}
& & Implementar el software para automatización de los casos de prueba. & \\
\cline{3-3}
& & Ejecutar el software para automatización de los casos de prueba. & \\
\hline
\end{tabular}}
\caption{Planificación de actividades del proyecto.}
\label{planificacion}
\end{sidewaystable}

