\chapter{Introducción}

Hoy en día muchas empresas de software, siguen procedimientos de desarrollo
ágiles, acelerando la producción y publicación de productos hacia el mercado;
esto presenta un gran reto tanto desde la perspectiva del aseguramiento de la
calidad del producto, como del seguimiento y control de la corrección de
errores.

La clave para agilizar estos procesos parte por optimizar y automatizar los
procesos de despliegue y evaluación del software, el cambio es necesario para
el manejo de proyectos de desarrollo grandes, o proyectos que siguen
lineamientos o estándares de cumplimiento muy estrictos.

Este capítulo define el propósito del presente trabajo, los antecedentes, los
objetivos, los alcances, y los factores que despertaron el interés por
resolverlos.

\section{Antecedentes}
\emph{Salesforce} es un servicio que apoya a la gestión de todo el flujo de
clientes que se tienen en una empresa, este tipo de software es conocido como
\emph{CRM}, y la meta principal es ayudar a gestionar las relaciones con los
clientes, además de conocer su necesidades y preferencias.

\emph{Salesforce} es el líder mundial como proveedor de su sistema \emph{CRM},
tanto para ventas, como servicio al cliente y mercadotecnia, por lo que es
necesario que provean a sus clientes de un software robusto y confiable.

Entre los muchos componentes que integran este servicio, uno de los elementos
básicos es aquel que gestiona los productos y las listas de precios. El módulo
de productos provee las funcionalidad de gestión para el catálogo de
los elementos que pueden comercializarse, mientras que el módulo de listas de
precios permite crear conjuntos personalizados de productos con precios de
lista asociados para usos específicos; ambos módulos estrechamente vinculados
entre si, y que son piezas fundamentales para otros componentes dentro del
sistema.

Garantizar el éxito del servicio implica tener un proceso de evaluación y
mejora continua del producto, haciendo que aquellos procesos repetitivos sean
automatizados y encaminados a cubrir aquellos criterios de calidad que
verdaderamente aporten valor al cliente.

\section{Definición del problema}
Siendo el módulo de productos y listas de precios componentes participes en la
funcionalidad provista por otros módulos de \emph{Salesforce}, un potencial
error en este podría tener un efecto en cascada con el potencial de afectar el
servicio completo, es por esta razón que es necesario priorizar la evaluación
de las funcionalidades disponibles por estos módulos.

\emph{Salesforce} sigue un política de actualizaciones muy continua, pudiendo
existir hasta tres versiones por año, lo que sin una automatización correcta
de las pruebas de interfaz de usuario, implicaría un enorme gasto de recursos
para la compañía.

Por lo mencionado se define el problema como:

\emph{«Garantizar la calidad de los elementos que componen la interfaz de
usuario requiere de un proceso de evaluación continuada y eficiente, de forma
que los módulos evaluados y sus diferentes versiones contengan la menor
cantidad posible de errores.»}.

\section{Objetivos}

\subsection{Objetivo General}
Implementar un \emph{framework} para la automatización de las pruebas de interfaz de
usuario en el módulo de Productos y Listas de Precios en \emph{Salesforce},
para garantizar un procedimiento continuo de evaluación y minimizar la cantidad
de errores que contiene el software.

\subsection{Objetivos Específicos}
\begin{itemize}
\item Formular los casos de prueba necesarios que los módulos de gestión de
    productos y listas de precios requieran para cubrir los atributos de calidad
    requeridos.
\item Diseñar e implementar los modelos y bibliotecas de funciones que
    conforman un \emph{framework} de automatización.
\item Automatizar los casos de prueba de las funciones que componen la interfaz
    de usuario del módulo de gestión de productos y listas de precios.
\end{itemize}

\section{Justificación}
Hoy en día, la automatización de pruebas es una tarea esencial para proporcionar
un servicio de evaluación de calidad adecuado, ya que los sistemas han crecido
tanto en tamaño como en complejidad.

Con una estrategia de automatización implementada apropiadamente el tiempo de
evaluar las nuevas funcionalidades del sistema, además de la evaluación de las
funcionalidades antiguas será reducido drásticamente, consiguiendo mejorar la
cobertura de las pruebas, junto con los beneficios propios de la automatización
\parencite{Software}.

\section{Alcance}
El proyecto cubrirá exclusivamente los componentes de interfaz del módulo de
productos y listas de precios, sin considerar aquellas funcionalidades dentro de
estos que requieran la utilización de otros módulos, de esta forma no se tomarán
en cuenta los siguientes aspectos:

\begin{itemize}
\item Generación de gráficas dentro de los módulos a evaluar.
\item Configuración de múltiples familias de productos.
\item Gestión de permisos de visibilidad de los elementos hacia diversas
    categorías de usuarios.
\item Evaluación y cumplimiento de los permisos de usuarios anteriormente
    citados.
\end{itemize}

