\chapter{HTSM}

\emph{HTSM (Heuristic Test Strategy Model)} fue creado por James Bach en 1996
para que lo usen los evaluadores profesionales como una colección estructurada
de recordatorios de qué pensar cuando están creando pruebas, divide el
pensamiento de la creación de pruebas en diferentes ejes de análisis, que,
cuando se unen, permiten al probador crear una estrategia de prueba
holística\cite{Bach}.

En este capitulo desglosaremos, y describiremos todos sus componentes, de tal
forma que este construya un marco solido para la planificación, y ejecución de
las pruebas de calidad del producto.

\section{Entorno de proyecto}
\subsection{Misión}

La misión del proyecto es:

«Evaluar la calidad de las funcionalidades provistas por \emph{Salesforce} que
componen el modulo de gestión de productos y listas de precios, con el diseño y
ejecución de multiples tecnicas de evaluacion, para que de esta manera se pueda
garantizar la calidad del producto para los clientes».

\subsection{Fuentes de Información}

Para evaluar los componentes antes citados, se encontraron las siguientes
fuentes de información respecto al producto:

\begin{description}
\item [Centro de Ayuda] Salesforce ofrece un amplio conjunto de documentación,
información general, preguntas frecuentes, y contacto con el servicio de
asistencia técnica desde su sitio de ayuda (https://help.salesforce.com/).

Estos recursos serán utiles para conocer las reclamaciones de los usuarios, las
caracteristicas criticas del producto, y las estrategias del fabricante hacia
sus clientes.

\item [Centro de Desarrollo] Salesforce tambien posee un sitio web
especificamente para compartir recursos de desarrollo sobre la plataforma
(https://developer.salesforce.com/).

Este sitio se prodrá aprovechar para consultar las referencias a las API del
servicio, conocer acerca de los componentes y como pueden aprovecharse desde la
perspectiva del desarrollador.

\item [Recursos para administradores] Sitio web enfocado a ofrecer experiencias,
videos, herramientas, y un sin fin de recursos orientados a usuarios con el rol
de administración de recursos sobre la plataforma
(https://admin.salesforce.com/resources).

\item [Comunidad \emph{Trailblazer}] Sitio web enfocado a conectar a miembros de
la comunidad Salesforce, para compartir experiencias, aprender, y proveer de
nuevas ideas sobre la utilización del servicio
(https://success.salesforce.com/).

\end{description}

\subsection{Equipamiento}
\subsubsection{Software}
\subsubsection{Herramientas}
%\subsection{Cronograma}
%\subsection{Elementos a evaluar}
%\subsection{Documentos entregables}
%\section{Elementos del producto}
%\section{Criterios de calidad}
%\subsection{Confiabilidad}
%\subsection{Usabilidad}
%\subsection{Compatibilidad}
%\subsection{Soportabilidad}
%\subsection{Localizabilidad}
%\section{Técnicas de prueba}
%\subsection{Pruebas funcionales}
%\subsection{Pruebas de dominio}
%\subsection{Pruebas de rendimiento}
%\subsection{Pruebas de usuario}
%\subsection{Pruebas de aceptación}
%\subsection{Pruebas de localización}

