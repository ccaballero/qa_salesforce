\chapter{Introducción}

\emph{Salesforce} es una compañia americana conocida por producir
\emph{Sales Cloud} un CRM (\emph{Customer relationship management}), es decir,
un software para la administración de la relación con los clientes. Y de entre
los elementos que componen este software, uno de los puntos fundamentales es la
administración de sus productos, y sus precios.

En este documento se detalla todo el proceso de planificación, diseño, y
ejecución de casos de prueba para garantizar los atributos de calidad deseados
en el componente de gestión de productos, precios, y listas de precios. Tambien
se definen los problemas, los objetivos fundamentales, y los
factores que despertaron el interés por resolverlos.

\section{Antecedentes}

\section{Definición del problema}

Por lo mencionado se define el problema como:

\emph{“El componente de gestión de productos y precios en la plataforma Sales
Cloud, siendo un componente fundamental para el exito del producto requiere”}.

\section{Objetivos}

\subsection{Objetivo General}
Promover el intercambio de información entre los estudiantes, mediante el uso
de una red social para mejorar los métodos de adquisición del conocimiento.

\subsection{Objetivos Específicos}
\begin{itemize}
\item Agilizar la creación de espacios virtuales para incrementar la cantidad y
variabilidad de estos.
\item Facilitar el intercambio de recursos entre los estudiantes para acelerar
la adquisición de experiencia.
\item Mejorar los canales de comunicación entre estudiantes y docentes para
facilitar la retroalimentación.
\item Planear estrategias que fomenten la participación para mantener activo el
sistema.
\end{itemize}

\section{Justificación}

\section{Alcance}
Es necesario mencionar que escapan de las funciones de este sistema la
interacción entre el sistema desarrollado y otras redes sociales, sea para
provisión o consumo de recursos.

Otra restricción impuesta será el registro cerrado para usuarios, esta será
exclusivamente por medio de invitaciones, todo esto para crear una red social
de conexiones lo menos dispersas posibles.

