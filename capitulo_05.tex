\chapter{Conclusiones y Recomendaciones}

Este capítulo compendia la conclusión misma del proyecto, para finalizar
presentando las recomendaciones finales que se han planteado a partir de los
resultados obtenidos.

\section{Conclusiones}
Todos los objetivos fueron cubiertos correctamente, como se describen a
continuación:

\begin{itemize}
\item El \textbf{primer objetivo} (Formular los casos de prueba necesarios que
los módulos de gestión de productos y listas de precios requieran para cubrir
los atributos de calidad requeridos) ha sido alcanzado, ya que se consiguió un
buen grado de cobertura tanto del modulo de productos, como del modulo de listas
de precios con los casos de prueba planteados, y categorizando apropiadamente
según el uso apropiado que requiera, ya sea para la ejecución de una prueba de
regresión, una de aceptación o la evaluación total del software.
\item El \textbf{segundo objetivo} (Diseñar e implementar los modelos y
bibliotecas de funciones que conforman un \emph{framework} de automatización)
ha sido cubierto, ya que el \emph{framework} de automatización esta utilizando
los patrones de diseño apropiados, e implementados con una biblioteca de
automatización ampliamente utilizada, haciendo que pueda ser fácilmente
mantenible, usable y extensible.
\item El \textbf{tercer objetivo} (Automatizar los casos de prueba de las
funciones que componen la interfaz de usuario del módulo de gestión de productos
y listas de precios) ha sido cubierto, ya que se tienen implementados los casos
de prueba planteados.
\end{itemize}

Con lo que se puede concluir que el objetivo general a sido cumplido, ya que el
procedimiento de evaluación del modulo de Productos y Listas de Precio ahora
puede hacerse de forma continua y acomodandose a diferentes etapas del
desarrollo del software, minimizando la cantidad de posibles errores y mejorando
la calidad general del sistema.

\section{Recomendaciones}
Si bien se han cubierto las pruebas funcionales, además de realizarse pruebas
de dominio para los formularios y pruebas negativas para los mensajes de error,
se cree necesario extender para incluir dos tipos de pruebas que podrían ser
muy útiles para el software a evaluar.

\begin{itemize}
\item Pruebas de localización (l10n).
\item Pruebas de internacionalización (i18n).
\end{itemize}

Ambas muy necesarias para un software con soporte para múltiples idiomas,
culturas y formatos diversos.

