\chapter{Configuración y puesta en marcha del \emph{framework}}
\label{appendix_execution}

En este apéndice se describen los pasos necesarios para la ejecución del
\emph{framework}. Se comenzará con los detalles de la obtención del código
fuente, y las configuraciones previas a la ejecución, posteriormente se
demostrará como hacer la ejecución en diferentes contextos de prueba.

\section{Requerimientos de software}
El \emph{framework} desarrollado tiene tres ambientes de ejecución posibles:

\begin{description}
\item [Local:] Ejecución sobre el host donde se invocan los comandos.
\item [Docker:] Ejecución sobre contenedores de \emph{Docker}.
\item [BrowserStack:] Ejecución sobre el servicio \emph{BrowserStack}.
\end{description}

Para la ejecución del \emph{framework} se debe tener instalado el software
detallado en el \emph{cuadro \ref{requirements}}, según el tipo de contexto
sobre el cual va a ser ejecutado.

\begin{table}[H]
\centering
\begin{tabular}{|l|c|c|c|}
\hline
\footnotesize{\textbf{Software}} &
\footnotesize{\textbf{Local}} &
\footnotesize{\textbf{Docker}} &
\footnotesize{\textbf{BrowserStack}} \\
\hline
\footnotesize{\emph{Git}} &
\footnotesize{versión 2.0 o superior} &
\footnotesize{versión 2.0 o superior} &
\footnotesize{versión 2.0 o superior} \\
\hline
\footnotesize{\emph{Node.js}} &
\footnotesize{versión 9.11 o superior} &
\footnotesize{versión 9.11 o superior} &
\footnotesize{versión 9.11 o superior} \\
\hline
\footnotesize{\emph{Docker}} &
\footnotesize{No necesario} &
\footnotesize{versión 18 o superior} &
\footnotesize{No necesario} \\
\hline
\end{tabular}
\caption{Requerimientos de software para la ejecución.}
\label{requirements}
\source{Elaboración propia.}
\end{table}

\section{Código fuente}
Todo el código fuente del \emph{framework} se encuentra versionado y disponible
desde \emph{github}, en la siguiente dirección:
\\
\\
\centerline{\textbf{https://github.com/ccaballero/salesforce\_automation}}

\section{Obtención del código y configuración}
Para la obtención del código se debe proceder a la clonación del repositorio,
desde una terminal del sistema operativo como se muestra en la
\emph{figura \ref{clone}}.

\begin{figure}[H]
\footnotesize
\begin{verbatim}
$ git clone https://github.com/ccaballero/salesforce_automation
Clonando en 'salesforce_automation'...
remote: Enumerating objects: 481, done.
remote: Counting objects: 100% (481/481), done.
remote: Compressing objects: 100% (283/283), done.
remote: Total 481 (delta 306), reused 358 (delta 185), pack-reused 0
Recibiendo objetos: 100% (481/481), 83.17 KiB | 166.00 KiB/s, listo.
Resolviendo deltas: 100% (306/306), listo.

$ cd salesforce_automation
\end{verbatim}
\caption{Obtención del código fuente.}
\label{clone}
\source{Elaboración propia.}
\end{figure}

Posteriormente a clonar el repositorio, deben instalarse las bibliotecas
dependientes del \emph{framework}, como se muestra en la
\emph{figura \ref{npm}}.

\begin{figure}
\footnotesize
\begin{verbatim}
~/salesforce_automation $ npm install

> fibers@3.1.1 install ~/salesforce_automation/node_modules/fibers
> node build.js || nodejs build.js

make: se entra en el directorio '~/salesforce_automation/node_modules/fibers/build'
  CXX(target) Release/obj.target/fibers/src/fibers.o
../src/fibers.cc: En la función ‘void uni::SetAccessor(v8::Isolate*, v8::Local<v8::Object>, 
v8::Local<v8::String>, uni::FunctionType (*)(v8::Local<v8::String>, const GetterCallbackInfo
&), void (*)(v8::Local<v8::String>, v8::Local<v8::Value>, const SetterCallbackInfo&))’:
../src/fibers.cc:341:87: aviso: cast between incompatible function types from ‘uni::Function
Type (*)(v8::Local<v8::String>, const GetterCallbackInfo&)’ {aka ‘void (*)(v8::Local<v8::Str
ing>, const v8::PropertyCallbackInfo<v8::Value>&)’} to ‘v8::AccessorNameGetterCallback’ {aka
‘void (*)(v8::Local<v8::Name>, const v8::PropertyCallbackInfo<v8::Value>&)’} [-Wcast-functio
n-type]
   object->SetAccessor(isolate->GetCurrentContext(), name, (AccessorNameGetterCallback)gette
r, (AccessorNameSetterCallback)setter).ToChecked();
                                                                                       ^~~~~
In file included from ../src/coroutine.h:1,
                 from ../src/fibers.cc:1:
../src/fibers.cc: En el ámbito global:
~/.node-gyp/11.14.0/include/node/node.h:544:43: aviso: cast between incompatible function ty
pes from ‘void (*)(v8::Handle<v8::Object>)’ {aka ‘void (*)(v8::Local<v8::Object>)’} to ‘node
::addon_register_func’ {aka ‘void (*)(v8::Local<v8::Object>, v8::Local<v8::Value>, void*)’} 
[-Wcast-function-type]
       (node::addon_register_func) (regfunc),                          \
                                           ^
~/.node-gyp/11.14.0/include/node/node.h:578:3: nota: en expansión de macro ‘NODE_MODULE_X’
   NODE_MODULE_X(modname, regfunc, NULL, 0)  // NOLINT (readability/null_usage)
   ^~~~~~~~~~~~~
../src/fibers.cc:916:1: nota: en expansión de macro ‘NODE_MODULE’
 NODE_MODULE(fibers, init)
 ^~~~~~~~~~~
  CXX(target) Release/obj.target/fibers/src/coroutine.o
  CC(target) Release/obj.target/fibers/src/libcoro/coro.o
  SOLINK_MODULE(target) Release/obj.target/fibers.node
  COPY Release/fibers.node
make: se sale del directorio '~/salesforce_automation/node_modules/fibers/build'
Installed in `~/salesforce_automation/node_modules/fibers/bin/linux-x64-67-glibc/fibers.node`
npm notice created a lockfile as package-lock.json. You should commit this file.
added 248 packages from 819 contributors and audited 682 packages in 53.763s
found 0 vulnerabilities
\end{verbatim}
\caption{Instalación de bibliotecas del proyecto.}
\label{npm}
\source{Elaboración propia.}
\end{figure}

Ya instaladas las bibliotecas requeridas por el \emph{framework}, se deben
configurar las credenciales a utilizarse para acceder al servicio de
\emph{Salesforce}; para eso es necesario copiar el archivo \emph{config.dist.js}
sobre el mismo directorio, con el nombre \emph{config.js} y agregar los datos de
las credenciales en su interior, como se muestra en la \emph{figura
\ref{config}}.

Para obtener las credenciales, es necesario crear una cuenta de desarrollador en
la plataforma \emph{Salesforce} y configurarla para evitar la solicitud de doble
factor de autenticación.

De la misma manera se requieren las credenciales de \emph{BrowserStack}, si es
que se desea ejecutar el framework sobre esta plataforma.

\begin{figure}[H]
\footnotesize
\begin{verbatim}
module.exports={
    timeout:40000
  , url:{
      login:'https://login.salesforce.com'
    }
  , credentials:{
        administrator:{
            username:'<nombre de usuario salesforce>'
          , password:'<contraseña del usuario salesforce>'
        }
    }
  , browserstack:{
        user:'<nombre de usuario browserstack>'
      , key:'<clave provista por browserstack>'
      , capabilities:[{
            os:'OS X'
          , os_version:'Mojave'
          , browserName:'Chrome'
          , browser_version:'72.0 beta'
        }]
    }
  , docker:{
        host:'127.0.0.1'
      , port:4444
    }
};
\end{verbatim}
\caption{Contenido del fichero config.js.}
\label{config}
\source{Elaboración propia.}
\end{figure}

El framework consta de múltiples grupos de prueba ya especificados en el
capitulo de desarrollo del \emph{frawework}, por lo que pueden ejecutarse según
estos, o puntualizando el caso de prueba que quiere ejecutarse.

\section{Ejecución de casos de prueba}
Para la ejecución de casos de prueba, se tienen disponibles tres
configuraciones, uno por cada contexto en el que se tiene la posibilidad de
ejecución, estas únicamente difieren según el archivo de configuración que
deberá utilizarse, como se muestra en la \emph{figura \ref{execution}}.

\begin{figure}[H]
\footnotesize
\begin{verbatim}
# Ejecución en host local del caso de prueba A001
~/salesforce_automation $ ./node_modules/.bin/wdio wdio.standalone.conf.js --suite A001

# Ejecución en host local de todos los casos de prueba de tipo dominio
~/salesforce_automation $ ./node_modules/.bin/wdio wdio.standalone.conf.js --suite domain

# Ejecución en host local de todos los casos de prueba relacionados con productos
~/salesforce_automation $ ./node_modules/.bin/wdio wdio.standalone.conf.js --suite products
\end{verbatim}
\caption{Ejemplos de ejecución de casos de prueba.}
\label{execution}
\source{Elaboración propia.}
\end{figure}

