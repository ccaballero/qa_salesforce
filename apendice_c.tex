\chapter{Glosario}\label{appendix_glossary}
A continuación se detallan algunos términos utilizados a lo largo del
documento, para que esta sección sirva como una referencia global del documento.

\begin{description}
\item [\emph{API}:] Sigla que procede de la lengua inglesa y que alude a la
    expresión \emph{Application Programming Interface} (cuya traducción es
    interfaz de programación de aplicaciones). El concepto hace referencia a los
    procesos, las funciones y los métodos que brinda una determinada biblioteca
    de programación a modo de capa de abstracción para que sea empleada por otro
    programa informático.
\item [\emph{AUT}:] Sigla que procede de la lengua inglesa y que alude a la
    expresión \emph{Application under Test} (cuya traducción es aplicación bajo
    pruebas). Después de la fase de diseño y codificación en el ciclo de vida
    del desarrollo del software, la aplicación viene para la prueba y en ese
    momento la aplicación se declara como Aplicación bajo prueba.
\item [\emph{BDD}:] Sigla que procede de la lengua inglesa y que alude a la
    expresión \emph{Behavior-driver development} (cuya traducción es desarrollo
    orientado a comportamiento). Es una metodología ágil de desarrollo de
    software en la que una aplicación está documentada y diseñada en torno al
    comportamiento que un usuario espera experimentar al interactuar con ella.
\item [\emph{Boundary-value Analysis}:] El análisis de valor de límite es una
    técnica para probar el valor de límite entre particiones válidas e inválidas
    en el diseño de casos de prueba
\item [\emph{CRM}:] Sigla que procede de la lengua inglesa y que alude a la
    expresión \emph{Customer Relationship Management} (cuya traducción es
    gestión de las relaciones con clientes). Es una tecnología para gestionar
    todas las relaciones e interacciones de su empresa con clientes y clientes
    potenciales. El objetivo es simple: mejorar las relaciones comerciales. Un
    sistema de CRM ayuda a las empresas a mantenerse conectadas con los
    clientes, racionalizar los procesos y mejorar la rentabilidad.
\item [\emph{Framework}:] Es un conjunto estandarizado de conceptos, prácticas y
    criterios para enfocar un tipo de problemática particular que sirve como
    referencia, para enfrentar y resolver nuevos problemas de índole similar. En
    el desarrollo de software, un entorno de trabajo es una estructura
    conceptual y tecnológica de asistencia definida, normalmente, con artefactos
    o módulos concretos de software, que puede servir de base para la
    organización y desarrollo de software. Típicamente, puede incluir soporte de
    programas, bibliotecas, y un lenguaje interpretado, entre otras
    herramientas, para así ayudar a desarrollar y unir los diferentes
    componentes de un proyecto.
\item [\emph{Github}:] Es una plataforma de colaboración y control de versiones
    basada en web para desarrolladores de software. Se utiliza para almacenar el
    código fuente de un proyecto y hacer un seguimiento del historial completo
    de todos los cambios en ese código.
\item [\emph{Javascript}:] Es el nombre de un lenguaje de programación; es
    decir, un lenguaje formal que brinda instrucciones a una computadora
    (ordenador) para generar ciertos datos. Se utiliza sobre todo para producir
    recursos interactivos en una página web. Por sus características,
    \emph{JavaScript} es un lenguaje imperativo, basado en prototipos y
    orientado a objetos. Por lo general se emplea del lado del cliente (lo que
    se conoce como \emph{client-side}), aunque también hay una forma de este
    lenguaje del lado del servidor (\emph{server-side}).
\item [\emph{Node.js}:] Es es un entorno en tiempo de ejecución multiplataforma,
    de código abierto, para la capa del servidor (pero no limitándose a ello)
    basado en el lenguaje de programación \emph{ECMAScript}, asíncrono, con I/O
    de datos en una arquitectura orientada a eventos y basado en el motor
    \emph{V8} de \emph{Google}.
\item [\emph{Page Object Model}:] Es un patrón de diseño que se ha hecho popular
    en la automatización de pruebas para mejorar el mantenimiento de las pruebas
    y reducir la duplicación de código. Un \emph{page object} es una clase
    orientada a objetos que sirve como interfaz para una página del \emph{AUT}.
\item [\emph{End-to-End Testing}:] Es una metodología utilizada para probar si
    el flujo de una aplicación se está ejecutando según lo diseñado de principio
    a fin. El propósito de llevar a cabo pruebas de extremo a extremo es
    identificar las dependencias del sistema y garantizar que la información
    correcta se transmita entre varios componentes del sistema.
\item [\emph{Salesforce}:] Compañía global de computación en la nube y software
    basada en la web, más conocida por su producto de gestión de relaciones con
    el cliente (\emph{CRM}). Para ayudar a los usuarios a manejar todas sus
    necesidades comerciales, como gestionar campañas de marketing, analizar el
    rendimiento y realizar un seguimiento de los gastos y las ventas.
\item [\emph{Selenium}:] Framework portátil para probar aplicaciones web. Las
    pruebas se pueden ejecutar en la mayoría de los navegadores web modernos.
    \emph{Selenium} se implementa en plataformas \emph{Windows}, \emph{Linux} y
    \emph{macOS}. Es un software de código abierto.
\item [\emph{TDD}:] Sigla que procede de la lengua inglesa y que alude a la
    expresión \emph{Test-driver development} (cuya traducción es desarrollo
    orientado a pruebas). Es un enfoque de desarrollo de software en el que se
    escribe una prueba antes de escribir el código fuente. Una vez que el nuevo
    código pasa la prueba, se refactoriza a un estándar aceptable. \emph{TDD} se
    asegura de que el código fuente sea probado en su totalidad y conduzca a un
    código modular, flexible y extensible. Se enfoca en escribir solo el código
    necesario para pasar las pruebas, haciendo que el diseño sea simple y claro.
\end{description}

