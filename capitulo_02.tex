\chapter{HTSM}

\emph{HTSM (Heuristic Test Strategy Model)}, es un conjunto de patrones
utilizados para diseñar una estrategia de pruebas para un producto en
particular.

El \emph{HSTM} fue creado por James Bach en 1996 para que lo usen los
evaluadores profesionales como una colección estructurada de recordatorios de
qué pensar cuando están creando pruebas, divide el pensamiento de la
creación de pruebas en diferentes ejes de análisis, que, cuando se unen,
permiten al probador crear una estrategia de prueba holística.

\section{Entorno de proyecto}
\subsection{Misión}
\subsection{Fuentes de Información}
\subsection{Equipamiento}
\subsubsection{Hardware}
\subsubsection{Software}
\subsubsection{Herramientas}
\subsection{Cronograma}
\subsection{Elementos a evaluar}
\subsection{Documentos entregables}
\section{Elementos del producto}
\section{Criterios de calidad}
\subsection{Confiabilidad}
\subsection{Usabilidad}
\subsection{Compatibilidad}
\subsection{Soportabilidad}
\subsection{Localizabilidad}
\section{Técnicas de prueba}
\subsection{Pruebas funcionales}
\subsection{Pruebas de dominio}
\subsection{Pruebas de rendimiento}
\subsection{Pruebas de usuario}
\subsection{Pruebas de aceptación}
\subsection{Pruebas de localización}

