\chapter{Glosario}\label{appendix_glossary}
A continuación se detallan algunos terminos utilizados a lo largo del
documento, para que esta sección sirva como una referencia global del documento.

\begin{description}
\item [API:] Sigla que procede de la lengua inglesa y que alude a la expresión
    \emph{Application Programming Interface} (cuya traducción es interfaz de
    programación de aplicaciones). El concepto hace referencia a los procesos,
    las funciones y los métodos que brinda una determinada biblioteca de
    programación a modo de capa de abstracción para que sea empleada por otro
    programa informático.
\item [AUT:] Sigla que procede de la lengua inglesa y que alude a la expresión
    \emph{Application under Test} (cuya traducción es aplicación bajo pruebas).
    Después de la fase de diseño y codificación en el ciclo de vida del
    desarrollo del software, la aplicación viene para la prueba y en ese momento
    la aplicación se declara como Aplicación bajo prueba.
\item [BDD:] Sigla que procede de la lengua inglesa y que alude a la expresión
    \emph{Behavior-driver development} (cuya traducción es desarrollo orientado
    a comportamiento). Es una metodología agil de desarrollo de software en la
    que una aplicación está documentada y diseñada en torno al comportamiento
    que un usuario espera experimentar al interactuar con ella.
\item [CRM:] Sigla que procede de la lengua inglesa y que alude a la expresión
    \emph{Customer Relationship Management} (cuya traducción es gestión de las
    relaciones con clientes). Es una tecnología para gestionar todas las
    relaciones e interacciones de su empresa con clientes y clientes
    potenciales. El objetivo es simple: mejorar las relaciones comerciales. Un
    sistema de CRM ayuda a las empresas a mantenerse conectadas con los
    clientes, racionalizar los procesos y mejorar la rentabilidad.
\item [Framework:]
\item [Github:]
\item [Javascript:]
\item [Node.js:]
\item [Page Object:]
\item [Pruebas End-to-End:]
\item [Salesforce:]
\item [Selenium:]
\item [Tabla de Myers:]
\item [TDD:]
\item [Webdriver.io:]
\end{description}

