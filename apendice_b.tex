\chapter{Reportes de error}\label{appendix_bugreport}
Este apéndice tiene como objetivo presentar los errores encontrados en los casos
de prueba que fueron evaluados en cada caso. Existen cuatro casos de prueba que
presentaron fallos tras la ejecución, como se describen en el
\emph{cuadro \ref{bugs}}.

\begin{table}
\centering
\begin{tabular}{|c|p{6.5cm}|c|c|c|}
\hline
\footnotesize{\textbf{Código}} & \footnotesize{\textbf{Caso de Prueba}} &
\footnotesize{\textbf{Resultado}} & \footnotesize{\textbf{Especificación}} \\
\hline
\footnotesize{\emph{A007}} &
\footnotesize{Precio de producto para una lista de precios es registrado después
de accionado el botón «Guardar»} &
\footnotesize{Bloqueado} &
\footnotesize{\emph{Cuadro \ref{tca007}}}
\\
\footnotesize{\emph{F041}} &
\footnotesize{Mensaje «Se guardó Entrada del catálogo de precios ."» se muestra
después de modificado un precio} &
\footnotesize{Bloqueado} &
\footnotesize{\emph{Cuadro \ref{tcf041}}}
\\
\footnotesize{\emph{D017}} &
\footnotesize{Formulario «Agregar a lista de precios» permite el registro,
cuando en el campo «Lista de precios» se ingresa el valor Nulo} &
\footnotesize{Fallido} &
\footnotesize{\emph{Cuadro \ref{tcd017}}}
\\
\footnotesize{\emph{D018}} &
\footnotesize{Formulario «Agregar a lista de precios» permite el registro,
cuando en el campo «Divisa» se ingresa el valor Nulo} &
\footnotesize{Fallido} &
\footnotesize{\emph{Cuadro \ref{tcd018}}}
\\
\hline
\end{tabular}
\caption{Casos de prueba que resultaron no exitosos.}
\label{bugs}
\source{Elaboración propia.}
\end{table}

Considerando que todos los casos de prueba fallidos comparten un problema común,
es posible englobar a todos en un solo caso de error, presentado en el
\emph{cuadro \ref{br}}.

\begin{table}
\renewcommand{\arraystretch}{1}
\linespread{1}
\centering
\begin{tabular}{|p{2.5cm}|p{2.8cm}|p{2.2cm}|p{2.8cm}|p{2.2cm}|}
\hline
\footnotesize{\textbf{Proyecto:}} &
\multicolumn{2}{c|}{\footnotesize{\emph{Salesforce} - Módulo de productos.}} &
\footnotesize{\textbf{Función:}} &
\multicolumn{1}{c|}{\footnotesize{[Ver-Relacionado]}} \\
\hline
\footnotesize{\textbf{ID:}} & \multicolumn{2}{c|}{\footnotesize{A007}} &
\footnotesize{\textbf{Prioridad:}} &
\multicolumn{1}{c|}{\footnotesize{Alta}} \\
\hline
\footnotesize{\textbf{Título:}} &
\multicolumn{4}{p{12.4cm}|}{\footnotesize{Precio de producto para una lista de
precios es registrado después de accionado el botón «Guardar»}} \\
\hline
\footnotesize{\textbf{Descripción:}} &
\multicolumn{4}{p{12.4cm}|}{\footnotesize{Verificar que una vez establecido un
precio estándar en un Producto, puede agregarse un precio nuevo en una lista de
precios creada previamente, desde el formulario provisto por el sistema desde
la pestaña «Detalles» en la vista de Producto.}} \\
\hline
\multirow{2}{*}{\footnotesize{\textbf{Requerimientos:}}} &
\footnotesize{\textbf{Software:}} &
\multicolumn{3}{p{7.8cm}|}{\footnotesize{Navegador \emph{Google Chrome}
versión 70.0.3538.110}} \\
\cline{2-5}
& \footnotesize{\textbf{Instrucciones de inicialización:}} &
\multicolumn{3}{p{7.8cm}|}{\footnotesize{
\vspace{-3mm}
\begin{enumerate}
\item Autenticarse en la plataforma \emph{Salesforce}.
\item Clic en el «Iniciador de Aplicaciones».
\item Clic en el enlace «Lista de precios».
\item Clic en el botón «Nuevo».
\item Rellenar el campo «Nombre de la lista de precios» con el valor
    \textbf{TESTA007}.
\item Clic en el botón «Guardar».
\item Clic en el «Iniciador de Aplicaciones».
\item Clic en el enlace «Productos».
\end{enumerate}
\vspace{-5mm}
}} \\
\hline
\footnotesize{\textbf{Pasos:}} &
\multicolumn{4}{p{11.8cm}|}{\footnotesize{
\vspace{-3mm}
\begin{enumerate}
\item Clic en el botón «Nuevo».
\item Rellenar el campo «Nombre del Producto» con el valor \textbf{TESTA007}.
\item Clic en el botón «Guardar».
\item Verificar que el mensaje enviado sea:
    \textbf{Producto TESTA007 ha sido creado}.
\item Clic en el pestaña «Relacionado».
\item Clic en el botón «Agregar precio estándar».
\item Rellenar el campo «Precio de la lista» con el valor \textbf{1}.
\item Clic en el botón «Guardar».
\item Clic en el botón «Agregar a lista de precios».
\item Seleccionar en el campo «Lista de precios» el valor \textbf{TESTA007}.
\item Clic en el botón «Siguiente».
\item Clic en el botón «Guardar».
\item Verificar que el mensaje enviado sea:
    \textbf{Se creó Entrada del catálogo de precios .}
\end{enumerate}
\vspace{-5mm}
}} \\
\hline
\multirow{2}{2.8cm}{\footnotesize{\textbf{Criterio de aceptación:}}} &
\footnotesize{\textbf{Resultado esperado:}} &
\multicolumn{3}{p{9.1cm}|}{\footnotesize{En la pestaña «Detalles», en la tabla
«Listas de precios» existen dos filas, una para el precio estándar, y otra para
la lista \textbf{TESTA007}.}} \\
\cline{2-5}
& \footnotesize{\textbf{Verificación:}} &
\multicolumn{3}{p{9.1cm}|}{\footnotesize{Comprobar elemento por elemento que los
valores de la tabla «Listas de precios» sean los valores anteriormente
rellenados.}} \\
\hline
\footnotesize{\textbf{Fecha:}} &
\multicolumn{1}{c|}{\footnotesize{2018-12-05}} &
\multicolumn{2}{l|}{\footnotesize{\textbf{Tiempo de ejecución:}}} &
\multicolumn{1}{c|}{\footnotesize{2 min 12 seg}} \\
\hline
\footnotesize{\textbf{Creado por:}} &
\multicolumn{4}{c|}{\footnotesize{CC - Carlos Caballero}} \\
\hline
\end{tabular}
\caption{Especificación del Caso de Prueba A007.}
\label{tca007}
\source{Elaboración propia.}
\end{table}

\begin{table}
\renewcommand{\arraystretch}{1}
\linespread{1}
\centering
\begin{tabular}{|p{2.5cm}|p{2.8cm}|p{2.2cm}|p{2.8cm}|p{2.2cm}|}
\hline
\footnotesize{\textbf{Proyecto:}} &
\multicolumn{2}{c|}{\footnotesize{\emph{Salesforce} - Módulo de productos.}} &
\footnotesize{\textbf{Función:}} &
\multicolumn{1}{c|}{\footnotesize{[Ver-Relacionado]}} \\
\hline
\footnotesize{\textbf{ID:}} & \multicolumn{2}{c|}{\footnotesize{F041}} &
\footnotesize{\textbf{Prioridad:}} &
\multicolumn{1}{c|}{\footnotesize{Media}} \\
\hline
\footnotesize{\textbf{Título:}} &
\multicolumn{4}{p{12.4cm}|}{\footnotesize{Mensaje «Se guardó Entrada del
catálogo de precios ."» se muestra después de modificado un precio}} \\
\hline
\footnotesize{\textbf{Descripción:}} &
\multicolumn{4}{p{12.4cm}|}{\footnotesize{Verificar que una vez establecido un
precio estándar en un Producto, al agregarse un precio nuevo en una lista de
precios creada previamente, se muestra el mensaje de éxito de la operación,
desde el formulario provisto por el sistema desde la pestaña «Detalles» en la
vista de Producto.}} \\
\hline
\multirow{2}{*}{\footnotesize{\textbf{Requerimientos:}}} &
\footnotesize{\textbf{Software:}} &
\multicolumn{3}{p{7.8cm}|}{\footnotesize{Navegador \emph{Google Chrome}
versión 70.0.3538.110}} \\
\cline{2-5}
& \footnotesize{\textbf{Instrucciones de inicialización:}} &
\multicolumn{3}{p{7.8cm}|}{\footnotesize{
\vspace{-3mm}
\begin{enumerate}
\item Autenticarse en la plataforma \emph{Salesforce}.
\item Clic en el «Iniciador de Aplicaciones».
\item Clic en el enlace «Lista de precios».
\item Clic en el botón «Nuevo».
\item Rellenar el campo «Nombre de la lista de precios» con el valor
    \textbf{TESTF041}.
\item Clic en el botón «Guardar».
\item Clic en el «Iniciador de Aplicaciones».
\item Clic en el enlace «Productos».
\end{enumerate}
\vspace{-5mm}
}} \\
\hline
\footnotesize{\textbf{Pasos:}} &
\multicolumn{4}{p{11.8cm}|}{\footnotesize{
\vspace{-3mm}
\begin{enumerate}
\item Clic en el botón «Nuevo».
\item Rellenar el campo «Nombre del Producto» con el valor \textbf{TESTF041}.
\item Clic en el botón «Guardar».
\item Verificar que el mensaje enviado sea:
    \textbf{Producto TESTF041 ha sido creado}.
\item Clic en el pestaña «Relacionado».
\item Clic en el botón «Agregar precio estándar».
\item Rellenar el campo «Precio de la lista» con el valor \textbf{1}.
\item Clic en el botón «Guardar».
\item Clic en el botón «Agregar a lista de precios».
\item Seleccionar en el campo «Lista de precios» el valor \textbf{TESTF041}.
\item Clic en el botón «Siguiente».
\item Clic en el botón «Guardar».
\end{enumerate}
\vspace{-5mm}
}} \\
\hline
\multirow{2}{2.8cm}{\footnotesize{\textbf{Criterio de aceptación:}}} &
\footnotesize{\textbf{Resultado esperado:}} &
\multicolumn{3}{p{9.1cm}|}{\footnotesize{Valor del precio del producto en la
lista de precio es agregado a la tabla de «Listas de precios».}} \\
\cline{2-5}
& \footnotesize{\textbf{Verificación:}} &
\multicolumn{3}{p{9.1cm}|}{\footnotesize{Verificar que el mensaje enviado sea:
\textbf{Se creó Entrada del catálogo de precios .}}} \\
\hline
\footnotesize{\textbf{Fecha:}} &
\multicolumn{1}{c|}{\footnotesize{2018-12-05}} &
\multicolumn{2}{l|}{\footnotesize{\textbf{Tiempo de ejecución:}}} &
\multicolumn{1}{c|}{\footnotesize{2 min 06 seg}} \\
\hline
\footnotesize{\textbf{Creado por:}} &
\multicolumn{4}{c|}{\footnotesize{CC - Carlos Caballero}} \\
\hline
\end{tabular}
\caption{Especificación del Caso de Prueba F041.}
\label{tcf041}
\source{Elaboración propia.}
\end{table}

\begin{table}
\renewcommand{\arraystretch}{1}
\linespread{1}
\centering
\begin{tabular}{|p{2.5cm}|p{2.8cm}|p{2.2cm}|p{2.8cm}|p{2.2cm}|}
\hline
\footnotesize{\textbf{Proyecto:}} &
\multicolumn{2}{c|}{\footnotesize{\emph{Salesforce} - Módulo de productos.}} &
\footnotesize{\textbf{Función:}} &
\multicolumn{1}{c|}{\footnotesize{[Ver-Relacionado]}} \\
\hline
\footnotesize{\textbf{ID:}} & \multicolumn{2}{c|}{\footnotesize{D017}} &
\footnotesize{\textbf{Prioridad:}} &
\multicolumn{1}{c|}{\footnotesize{Baja}} \\
\hline
\footnotesize{\textbf{Título:}} &
\multicolumn{4}{p{12.4cm}|}{\footnotesize{Formulario «Agregar a lista de
precios» permite el registro, cuando en el campo «Lista de precios» se ingresa
el valor Nulo}} \\
\hline
\footnotesize{\textbf{Descripción:}} &
\multicolumn{4}{p{12.4cm}|}{\footnotesize{Verificar que una vez establecido un
precio estándar en un Producto, en el formulario «Agregar a lista de precios»,
el valor para el campo «Lista de precios» pasa al siguiente paso con el valor
Nulo.}} \\
\hline
\multirow{2}{*}{\footnotesize{\textbf{Requerimientos:}}} &
\footnotesize{\textbf{Software:}} &
\multicolumn{3}{p{7.8cm}|}{\footnotesize{Navegador \emph{Google Chrome}
versión 70.0.3538.110}} \\
\cline{2-5}
& \footnotesize{\textbf{Instrucciones de inicialización:}} &
\multicolumn{3}{p{7.8cm}|}{\footnotesize{
\vspace{-3mm}
\begin{enumerate}
\item Autenticarse en la plataforma \emph{Salesforce}.
\item Clic en el «Iniciador de Aplicaciones».
\item Clic en el enlace «Lista de precios».
\item Clic en el botón «Nuevo».
\item Rellenar el campo «Nombre de la lista de precios» con el valor
    \textbf{TESTD017}.
\item Clic en el botón «Guardar».
\item Clic en el «Iniciador de Aplicaciones».
\item Clic en el enlace «Productos».
\end{enumerate}
\vspace{-5mm}
}} \\
\hline
\footnotesize{\textbf{Pasos:}} &
\multicolumn{4}{p{11.8cm}|}{\footnotesize{
\vspace{-3mm}
\begin{enumerate}
\item Clic en el botón «Nuevo».
\item Rellenar el campo «Nombre del Producto» con el valor \textbf{TESTD017}.
\item Clic en el botón «Guardar».
\item Verificar que el mensaje enviado sea:
    \textbf{Producto TESTD017 ha sido creado}.
\item Clic en el pestaña «Relacionado».
\item Clic en el botón «Agregar precio estándar».
\item Rellenar el campo «Precio de la lista» con el valor \textbf{1}.
\item Clic en el botón «Guardar».
\item Clic en el botón «Agregar a lista de precios».
\item Seleccionar en el campo «Lista de precios» el valor: \textbf{Ninguno}.
\item Clic en el botón «Siguiente».
\end{enumerate}
\vspace{-5mm}
}} \\
\hline
\multirow{2}{2.8cm}{\footnotesize{\textbf{Criterio de aceptación:}}} &
\footnotesize{\textbf{Resultado esperado:}} &
\multicolumn{3}{p{9.1cm}|}{\footnotesize{El formulario «Agregar a lista de
precios», pasa al segundo paso.}} \\
\cline{2-5}
& \footnotesize{\textbf{Verificación:}} &
\multicolumn{3}{p{9.1cm}|}{\footnotesize{Verificar que el formulario se
encuentra en el segundo paso.}} \\
\hline
\footnotesize{\textbf{Fecha:}} &
\multicolumn{1}{c|}{\footnotesize{2018-12-05}} &
\multicolumn{2}{l|}{\footnotesize{\textbf{Tiempo de ejecución:}}} &
\multicolumn{1}{c|}{\footnotesize{2 min 01 seg}} \\
\hline
\footnotesize{\textbf{Creado por:}} &
\multicolumn{4}{c|}{\footnotesize{CC - Carlos Caballero}} \\
\hline
\end{tabular}
\caption{Especificación del Caso de Prueba D017.}
\label{tcd017}
\source{Elaboración propia.}
\end{table}

\begin{table}
\renewcommand{\arraystretch}{1}
\linespread{1}
\centering
\begin{tabular}{|p{2.5cm}|p{2.8cm}|p{2.2cm}|p{2.8cm}|p{2.2cm}|}
\hline
\footnotesize{\textbf{Proyecto:}} &
\multicolumn{2}{c|}{\footnotesize{\emph{Salesforce} - Módulo de productos.}} &
\footnotesize{\textbf{Función:}} &
\multicolumn{1}{c|}{\footnotesize{[Ver-Relacionado]}} \\
\hline
\footnotesize{\textbf{ID:}} & \multicolumn{2}{c|}{\footnotesize{D018}} &
\footnotesize{\textbf{Prioridad:}} &
\multicolumn{1}{c|}{\footnotesize{Baja}} \\
\hline
\footnotesize{\textbf{Título:}} &
\multicolumn{4}{p{12.4cm}|}{\footnotesize{Formulario «Agregar a lista de
precios» permite el registro, cuando en el campo «Divisa» se ingresa
el valor Nulo}} \\
\hline
\footnotesize{\textbf{Descripción:}} &
\multicolumn{4}{p{12.4cm}|}{\footnotesize{Verificar que una vez establecido un
precio estándar en un Producto, en el formulario «Agregar a lista de precios»,
el valor para el campo «Divisa» pasa al siguiente paso con el valor
Nulo.}} \\
\hline
\multirow{2}{*}{\footnotesize{\textbf{Requerimientos:}}} &
\footnotesize{\textbf{Software:}} &
\multicolumn{3}{p{7.8cm}|}{\footnotesize{Navegador \emph{Google Chrome}
versión 70.0.3538.110}} \\
\cline{2-5}
& \footnotesize{\textbf{Instrucciones de inicialización:}} &
\multicolumn{3}{p{7.8cm}|}{\footnotesize{
\vspace{-3mm}
\begin{enumerate}
\item Autenticarse en la plataforma \emph{Salesforce}.
\item Clic en el «Iniciador de Aplicaciones».
\item Clic en el enlace «Lista de precios».
\item Clic en el botón «Nuevo».
\item Rellenar el campo «Nombre de la lista de precios» con el valor
    \textbf{TESTD018}.
\item Clic en el botón «Guardar».
\item Clic en el «Iniciador de Aplicaciones».
\item Clic en el enlace «Productos».
\end{enumerate}
\vspace{-5mm}
}} \\
\hline
\footnotesize{\textbf{Pasos:}} &
\multicolumn{4}{p{11.8cm}|}{\footnotesize{
\vspace{-3mm}
\begin{enumerate}
\item Clic en el botón «Nuevo».
\item Rellenar el campo «Nombre del Producto» con el valor \textbf{TESTD018}.
\item Clic en el botón «Guardar».
\item Verificar que el mensaje enviado sea:
    \textbf{Producto TESTD018 ha sido creado}.
\item Clic en el pestaña «Relacionado».
\item Clic en el botón «Agregar precio estándar».
\item Rellenar el campo «Precio de la lista» con el valor \textbf{1}.
\item Clic en el botón «Guardar».
\item Clic en el botón «Agregar a lista de precios».
\item Seleccionar en el campo «Divisa» el valor: \textbf{Ninguno}.
\item Clic en el botón «Siguiente».
\end{enumerate}
\vspace{-5mm}
}} \\
\hline
\multirow{2}{2.8cm}{\footnotesize{\textbf{Criterio de aceptación:}}} &
\footnotesize{\textbf{Resultado esperado:}} &
\multicolumn{3}{p{9.1cm}|}{\footnotesize{El formulario «Agregar a lista de
precios», pasa al segundo paso.}} \\
\cline{2-5}
& \footnotesize{\textbf{Verificación:}} &
\multicolumn{3}{p{9.1cm}|}{\footnotesize{Verificar que el formulario se
encuentra en el segundo paso.}} \\
\hline
\footnotesize{\textbf{Fecha:}} &
\multicolumn{1}{c|}{\footnotesize{2018-12-05}} &
\multicolumn{2}{l|}{\footnotesize{\textbf{Tiempo de ejecución:}}} &
\multicolumn{1}{c|}{\footnotesize{2 min 01 seg}} \\
\hline
\footnotesize{\textbf{Creado por:}} &
\multicolumn{4}{c|}{\footnotesize{CC - Carlos Caballero}} \\
\hline
\end{tabular}
\caption{Especificación del Caso de Prueba D018.}
\label{tcd018}
\source{Elaboración propia.}
\end{table}

\begin{table}
\renewcommand{\arraystretch}{1}
\linespread{1}
\centering
\begin{tabular}{|p{2.5cm}|p{2.8cm}|p{2.2cm}|p{2.8cm}|p{2.2cm}|}
\hline
\footnotesize{\textbf{Proyecto:}} &
\multicolumn{2}{c|}{\footnotesize{\emph{Salesforce} - Módulo de productos.}} &
\footnotesize{\textbf{Función:}} &
\multicolumn{1}{c|}{\footnotesize{[Ver-Relacionado]}} \\
\hline
\footnotesize{\textbf{ID:}} & \multicolumn{2}{c|}{\footnotesize{BUG001}} &
\footnotesize{\textbf{Prioridad:}} &
\multicolumn{1}{c|}{\footnotesize{Media}} \\
\hline
\footnotesize{\textbf{Título:}} &
\multicolumn{4}{p{12.4cm}|}{\footnotesize{No es posible establecer el precio de
un producto para una lista de precios desde el formulario «Agregar a lista de
precios».}} \\
\hline
\footnotesize{\textbf{Descripción:}} &
\multicolumn{4}{p{12.4cm}|}{\footnotesize{Desde dentro de la pestaña
«Relacionado» en la vista de producto, y sobre el formulario «Agregar a lista de
precios», no existe ninguna posibilidad de rellenar valores, tales que el
formulario haga un registro apropiado.}} \\
\hline
\multirow{2}{*}{\footnotesize{\textbf{Requerimientos:}}} &
\footnotesize{\textbf{Software:}} &
\multicolumn{3}{p{7.8cm}|}{\footnotesize{Navegador \emph{Google Chrome}
versión 70.0.3538.110}} \\
\cline{2-5}
& \footnotesize{\textbf{Instrucciones de inicialización:}} &
\multicolumn{3}{p{7.8cm}|}{\footnotesize{
\vspace{-3mm}
\begin{enumerate}
\item Autenticarse en la plataforma \emph{Salesforce}.
\item Clic en el «Iniciador de Aplicaciones».
\item Clic en el enlace «Lista de precios».
\item Clic en el botón «Nuevo».
\item Rellenar el campo «Nombre de la lista de precios» con el valor
    \textbf{TESTBUG001}.
\item Clic en el botón «Guardar».
\item Clic en el «Iniciador de Aplicaciones».
\item Clic en el enlace «Productos».
\end{enumerate}
\vspace{-5mm}
}} \\
\hline
\footnotesize{\textbf{Pasos:}} &
\multicolumn{4}{p{11.8cm}|}{\footnotesize{
\vspace{-3mm}
\begin{enumerate}
\item Clic en el botón «Nuevo».
\item Rellenar el campo «Nombre del Producto» con el valor \textbf{TESTBUG001}.
\item Clic en el botón «Guardar».
\item Verificar que el mensaje enviado sea:
    \textbf{Producto TESTBUG001 ha sido creado}.
\item Clic en el pestaña «Relacionado».
\item Clic en el botón «Agregar precio estándar».
\item Rellenar el campo «Precio de la lista» con el valor \textbf{1}.
\item Clic en el botón «Guardar».
\item Clic en el botón «Agregar a lista de precios».
\item Seleccionar en el campo «Lista de precios» el valor: \textbf{Ninguno}.
\item Seleccionar en el campo «Divisa» el valor: \textbf{Ninguno}.
\item Clic en el botón «Siguiente».
\end{enumerate}
\vspace{-5mm}
}} \\
\hline
\footnotesize{\textbf{Resultado Actual:}} &
\multicolumn{4}{p{12.4cm}|}{\footnotesize{Se ven los mensajes de validación,
que indican que los valores no pueden ser nulos.}} \\
\hline
\footnotesize{\textbf{Resultado Esperado:}} &
\multicolumn{4}{p{12.4cm}|}{\footnotesize{Se muestren los componentes del
segundo paso del formulario, y puedan ingresarse los valores de precios en esta.
}} \\
\hline
\footnotesize{\textbf{Camino alternativo:}} &
\multicolumn{4}{p{12.4cm}|}{\footnotesize{Es posible establecer un precio para
un producto desde la vista de «Lista de Precios».
}} \\
\hline
\footnotesize{\textbf{Creado por:}} &
\multicolumn{2}{c|}{\footnotesize{CC - Carlos Caballero}} &
\footnotesize{\textbf{Fecha:}} &
\multicolumn{1}{c|}{\footnotesize{2018-12-05}} \\
\hline
\end{tabular}
\caption{Reporte de Error sobre el formulario «Agregar a lista de precios».}
\label{br}
\source{Elaboración propia.}
\end{table}

