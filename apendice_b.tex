\chapter{Reportes de error}\label{appendix_bugreport}
Este apendice tiene como objetivo presentar los errores encontrados en los casos
de prueba que fueron evaluados en cada caso. Existen cuatro casos de prueba que
presentaron fallos tras la ejecución, como se describen en el cuadro
\ref{bugs}.

\begin{table}[H]
\centering
\begin{tabular}{|c|p{6.5cm}|c|c|c|}
\hline
\footnotesize{\textbf{Código}} & \footnotesize{\textbf{Caso de Prueba}} &
\footnotesize{\textbf{Resultado}} & \footnotesize{\textbf{Especificación}} &
\footnotesize{\textbf{Reporte de Error}} \\
\hline
\footnotesize{\emph{A007}} &
\footnotesize{Precio de producto para una lista de precios es registrado después
de accionado el botón «Guardar»} &
\footnotesize{Bloqueado} &
\footnotesize{\emph{\ref{tca007}}} &
\footnotesize{\emph{\ref{bra007}}}
\\
\footnotesize{\emph{F041}} &
\footnotesize{Mensaje «Se guardó Entrada del catálogo de precios ."» se muestra
después de modificado un precio} &
\footnotesize{Bloqueado} &
\footnotesize{\emph{\ref{tcf041}}} &
\footnotesize{\emph{\ref{brf041}}}
\\
\footnotesize{\emph{D017}} &
\footnotesize{Formulario «Agregar a lista de precios» permite el registro,
cuando en el campo «Lista de precios» se ingresa el valor Nulo} &
\footnotesize{Fallido} &
\footnotesize{\emph{\ref{tcd017}}} &
\footnotesize{\emph{\ref{brd017}}}
\\
\footnotesize{\emph{D018}} &
\footnotesize{Formulario «Agregar a lista de precios» permite el registro,
cuando en el campo «Divisa» se ingresa el valor Nulo} &
\footnotesize{Fallido} &
\footnotesize{\emph{\ref{tcd018}}} &
\footnotesize{\emph{\ref{brd018}}}
\\
\hline
\end{tabular}
\caption{Casos de prueba que resultaron no exitosos.}
\label{bugs}
\source{Elaboración propia.}
\end{table}

\begin{table}[H]
\centering
\begin{tabular}{|p{2.5cm}|p{2.5cm}|p{2.8cm}|p{2.5cm}|p{2.6cm}|}
\hline
\footnotesize{\textbf{Proyecto:}} &
\multicolumn{2}{c|}{\footnotesize{\emph{Salesforce} - Módulo de productos.}} &
\footnotesize{\textbf{Función:}} & \footnotesize{[Ver - Relacionado]} \\
\hline
\footnotesize{\textbf{ID:}} & \multicolumn{2}{c|}{\footnotesize{A007}} &
\footnotesize{\textbf{Prioridad:}} & \footnotesize{3} \\
\hline
\footnotesize{\textbf{Título:}} &
\multicolumn{4}{p{12.4cm}|}{\footnotesize{Precio de producto para una lista de
precios es registrado después de accionado el botón «Guardar»}} \\
\hline
\footnotesize{\textbf{Descripción:}} &
\multicolumn{4}{p{12.4cm}|}{\footnotesize{.}} \\
\hline
\multirow{2}{*}{\footnotesize{\textbf{Requerimientos:}}} &
\footnotesize{\textbf{Software:}} &
\multicolumn{3}{p{7.8cm}|}{\footnotesize{Navegador \emph{Google Chrome}
version 70.0.3538.110}} \\
\cline{2-5}
& \footnotesize{\textbf{Instrucciones de inicialización:}} &
\multicolumn{3}{p{7.8cm}|}{\footnotesize{
\begin{enumerate}
\item Autenticarse en la plataforma \emph{Salesforce}.
\item Clic en el «Iniciador de Aplicaciones».
\item Clic en el enlace «Productos».
\end{enumerate}
}} \\
\hline
\footnotesize{\textbf{Pasos:}} &
\multicolumn{4}{p{11.8cm}|}{\footnotesize{
\begin{enumerate}
\item Clic en el icono de «Controles de Vista de Lista».
\item Clic en el boton «Nuevo».
\item Rellenar el campo «Nombre del Producto» con el valor \textbf{TESTA001}.
\item Clic en el boton «Guardar».
\item Verificar que el mensaje enviado sea: \textbf{Producto TESTA001 ha sido creado}.
\item Verificar que en la vista de Producto el «Título» sea: \textbf{TESTA001}.
\item Verificar que en la vista de Producto el subtítulo «Código de Producto» este vacio.
\item Verificar que en la vista de Producto el subtítulo «Familia de Producto» este vacio.
\item Clic en el boton «Detalles».
\end{enumerate}
}} \\
\hline
\multirow{2}{2.8cm}{\footnotesize{\textbf{Criterio de aceptación:}}} &
\footnotesize{\textbf{Resultado esperado:}} &
\multicolumn{3}{p{9.1cm}|}{\footnotesize{En la pestaña «Detalles», todos los
valores exceptuando el «Nombre de Producto» se encuentran con los valores
vacios.}} \\
\cline{2-5}
& \footnotesize{\textbf{Verificación:}} &
\multicolumn{3}{p{9.1cm}|}{\footnotesize{Comprobar elemento por elemento que los
valores se encuentren vacios, y que el campo «Nombre de Producto» sea igual a:
\textbf{TESTA001}.}} \\
\hline
\footnotesize{\textbf{Adjuntos:}} &
\multicolumn{4}{p{12.4cm}|}{\footnotesize{}} \\
\hline
\footnotesize{\textbf{Notas:}} &
\multicolumn{4}{p{12.4cm}|}{\footnotesize{}} \\
\hline
\footnotesize{\textbf{Fecha:}} & \footnotesize{2018-12-05} & \multicolumn{2}{c|}{\footnotesize{\textbf{Tiempo de ejecución:}}} & \footnotesize{1 min 24 seg} \\
\hline
\footnotesize{\textbf{Creado por:}} &
\multicolumn{4}{p{12.4cm}|}{\footnotesize{CC - Carlos Caballero}} \\
\hline
\end{tabular}
\caption{Especificación del Caso de Prueba A007.}
\label{lltc}
\source{Elaboración propia.}
\end{table}

